\documentclass[12pt]{article}

\usepackage{graphicx}
\usepackage{epstopdf}
\usepackage[english]{babel}
\usepackage[latin5]{inputenc}
\usepackage{hyperref}
\usepackage[left=3cm,top=3cm,right=3cm,nohead,nofoot]{geometry}
\usepackage{datenumber}

\begin{document}

\begin{center}
\Huge
Project: Enigma Machine and Turing-Welchman Bombe

\vspace{10mm}
\Large
Maria Camila REMOLINA GUTI�RREZ
\large
maria.remolina\_gutierrez@telecom-sudparis.eu

\vspace{5mm}
\Large
Advisor: Prof. Eric RENAULT

\vspace{5mm}
\normalsize
\today
\end{center}

\vspace{5mm}
\begin{abstract}
	This is the work proposal to follow in the course Project as part of the master M1 in Computer Science and Communication Networks at T�l�com SudParis. The goal is to implement the Enigma machine used in World War II, followed by the Bombe machine that breaks the cipher, created by Alan Turing and Gordon Welchman at Bletchley Park.
\end{abstract}

\section{Introduction}

%Introducci�n a la propuesta de Monograf�a. Debe incluir un breve resumen del estado del arte del problema a tratar. Tambi�n deben aparecer citadas todas las referencias de la bibliograf�a (a menos de que se citen m�s adelante, en los objetivos o metodolog�a, por ejemplo)

Aqu\'i texto.


\section{General Goal}

%Objetivo general del trabajo. Empieza con un verbo en infinitivo.

Aqu\'i texto.


\section{Specific Goals}

%Objetivos espec�ficos del trabajo. Empiezan con un verbo en infinitivo.

\begin{itemize}
	\item Objetivo 1
	\item Objetivo 2
	\item Objetivo 3
	\item ...
\end{itemize}

\section{Methodology}

%Exponer DETALLADAMENTE la metodolog�a que se usar� en la Monograf�a. 

%Monograf�a te�rica o computacional: �C�mo se har�n los c�lculos te�ricos? �C�mo se har�n las simulaciones? �Qu� requerimientos computacionales se necesitan? �Qu� espacios f�sicos o virtuales se van a utilizar?

%Monograf�a experimental: Recordar que para ser aprobada, los aparatos e insumos experimentales que se usar�n en la Monograf�a deben estar previamente disponibles en la Universidad, o garantizar su disponibilidad para el tiempo en el que se realizar� la misma. �Qu� montajes experimentales se van a usar y que material se requiere? �En qu� espacio f�sico se llevar�n a cabo los experimentos? Si se usan aparatos externos, �qu� permisos se necesitan? Si hay que realizar pagos a terceros, �c�mo se financiar� esto?

Aqu\'i texto.

\section{Work Schedule}

\begin{table}[htb]
	\begin{tabular}{|c|cccccccccccccccc| }
	\hline
	Tareas $\backslash$ Semanas & 1 & 2 & 3 & 4 & 5 & 6 & 7 & 8 & 9 & 10 & 11 & 12 & 13 & 14 & 15 & 16  \\
	\hline
	1 & X & X &   &   &   &   &   & X & X &   &   &   &   &   &   &   \\
	2 &   & X & X &   & X & X & X &   &   & X & X & X &   & X & X &   \\
	3 &   &   &   & X &   &   &   & X &   &   &   & X &   &   & X &   \\
	4 & X & X & X & X & X & X & X & X & X & X &   &   &   &   &   &   \\
	5 &   &   &   &   & X &   &   &   & X &   &   & X &   &   & X &   \\
	\hline
	\end{tabular}
\end{table}
\vspace{1mm}

\begin{itemize}
	\item Tarea 1: Descripci\'on de la tarea 1
	\item Tarea 2: Descripci\'on de la tarea 2
	\item Tarea 3: Descripci\'on de la tarea 3
	\item ...
\end{itemize}

\begin{thebibliography}{10}

\bibitem{Jerry} J. Banks. \textit{Discrete-Event System Simulation}. Fourth Edition. Prentice Hall International Series in Industrial and Systems Engineering, pg 86 - 116 y 219 - 235, (2005).

\bibitem{Bronner} P. Bronner, A. Strunz, C. Silberhorn \& J.P. Meyn. European Journal of Physics, \textbf{30}, 1189-1200, (2009).

\bibitem{LabInt} P. D�az \& N. Barbosa: \textit{Obtenci�n de n�meros aleatorios}. Informe final del curso Laboratorio Intermedio. Universidad de Los Andes, Bogot�, Colombia, (2012).

\bibitem{Stefannov} A. Stefanov , N. Gisin , O. Guinnard , L. Guinnard \& H. Zbinden. Journal of Modern Optics, \textbf{47}:4, 595-598, (2000).

\end{thebibliography}

\section*{Advisor Signature}
\vspace{1.5cm}

\section*{Student Signature}

\end{document} 